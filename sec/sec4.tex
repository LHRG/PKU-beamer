\section{Selection rules in the LS-coupling scheme}

\begin{frame}{Electric dipole selection rules}
\only<1,5-6>{
    \begin{block}{Single electron}
        From conservation laws and quantum mechanics calculations:
        \begin{equation*}
            \begin{cases}
                    \Delta j=0, \pm1 \quad &(j=0 \nleftrightarrow j'=0),\\
                    \Delta m_j=0, \pm1, \quad &(m_j=0 \nleftrightarrow m_{j'}=0 \text{ if } \Delta j=0),\\
                    \Delta l=\pm1,&\\
                    \Delta m_l=0, \pm1.&
            \end{cases}
        \end{equation*}
    \end{block}}
\only<2-4>{
    \uncover<2-4>{
    LS-coupling scheme:
    \begin{equation*}
        \begin{cases}
            \Delta J=0,\pm1\quad &(J=0\nleftrightarrow J'=0), \\
            \Delta M_J=0,\pm1\quad &(M_J=0\nrightarrow M_{J'}=0\text{ if }\Delta J=0), \\
            \text{Parity changes}, \\
            \Delta l=\pm1\quad &\text{One electron jump}, \\
            \Delta L=0^{\footnotesize 1},\pm1\quad & (L=0\nleftrightarrow L'=0), \\
            \Delta S=0^{\footnotesize 2}.&
        \end{cases}
    \end{equation*}}
    \uncover<3-4>{
        1.$\Delta L=0$ is possible in principle, but more than one electron must be excited to a high-energy state.}\\
    \uncover<4>{
        2.Exception: In the mercury atom, however, transitions with $\Delta S=1$ occur, such as $6\text{s}^2\ {}^1\mathrm{S}_0- 6\text{s}6\text{p }^3\mathrm{P} _1$, that gives a so-called intercombination line with a wavelength of 254 nm.}}
\only<5-6>{
    \uncover<5->{
    jj-coupling scheme (two electrons) :
    \begin{equation*}
        \begin{cases}
            \Delta j_1=0,\quad\Delta j_2=0,\pm1,\quad\text{or}&\Delta j_1=0,\pm1,\quad\Delta j_2=0,\\
            \Delta J=0,\pm1\quad (J=0\nleftrightarrow J'=0),
        \end{cases}
    \end{equation*}}
    \uncover<6->{
    In fact, many elements fall between these two extreme situations, and the selection rules on both sides are not strictly followed.}}
\end{frame}

\begin{frame}{Magnetic dipole selection rules}
\uncover<1->{
    \begin{block}{}
        According to the multipole expansion of electromagnetic interactions, the magnetic dipole interaction can be described as the interaction between magnetic moment and vector radius, with the coefficient being the first-order spherical harmonic function. \\
        Therefore, the interaction can be expressed as
        \begin{equation*}
            {H}^{\prime}\propto\cos\theta Y_{10}(\theta,\phi)\propto Y_{00}(\theta,\phi).
        \end{equation*}
    \end{block}}
\uncover<2->{
    Therefore, apart from having the same selection rules as electric dipole transitions, there are also angular momentum selection rules: 
    \begin{equation*}
        \Delta l=0.
    \end{equation*}}
\uncover<3->{
    Directly generalized to multi-electron atoms:
    \begin{equation*}
        \Delta n=0,
        \begin{cases}
            \Delta L=0, \quad &\Delta S=0, \\
            \Delta J=0, \pm1, \quad &\Delta M_J=0, \pm1.
        \end{cases}
    \end{equation*}
    }
\end{frame}